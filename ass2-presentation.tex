\documentclass[12pt]{beamer}
\title{IQ Link}
\subtitle{COMP1110/1140/1510/6710 major assignment}
\author[u5838800, u6127601]{Alex Smith (u5838800), Yicong Du (u6127601)}
\date{Week 11, 2016}

\usetheme{Rochester}
\usecolortheme{orchid}

\usepackage{pdflscape}
\usepackage{color}
\usepackage{listings}
\lstset{language=Java,
                basicstyle=\footnotesize\ttfamily,
                keywordstyle=\footnotesize\color{blue}\ttfamily, }
\usepackage{graphicx}
\DeclareGraphicsExtensions{.png}

\newcommand{\fitimagewidth}[1]{\includegraphics[width=\textwidth]{#1}}
\newcommand{\fitimageheight}[1]{\centering{\includegraphics[height=\textheight]{#1}}}

\hypersetup{pdfstartview={Fit}}
\setbeamertemplate{navigation symbols}{}
\begin{document}

%Your presentation pdf file must be suitable for viewing on the lab projectors, and must include the following:
%- Names of all students who contributed.
%- A summary of what you did.
%- The class diagram as part of a design section.
%- Screen shots of your game.

%\section{Intro}
\frame{\titlepage}
\begin{frame}{Table of Contents}
	\tableofcontents[currentsection]
\end{frame}

\section{Summary}
\begin{frame}{Summary}
% Talk briefly about each point.
	\begin{itemize}
	\item Created Piece class with enumerators
		% Type, Segment, Orientation, Location
	\item Added placement methods
		% is_Valid, {get,update}PegsForPiecePlacement
	\item Implemented Viewer with one, then multiple, piece viewer
		% I did one, Yicong finished multiple
	\item Implemented full GUI for game
		\begin{itemize}
		\item Piece drawing and movement
			% Piece class, movement, two iterations of snapping
		\item Game selection
			% Welcome and main screen, then victory
		\item Hint viewer as string, then image
			% Useless little text box, then faded pieces
		\item Other QoL additions
			% Music, dynamic hints after a certain point?, 
		\end{itemize}
	\end{itemize}
\end{frame}

\section{Design}
\begin{frame}
	\centering{\Huge Design}
\end{frame}
\begin{frame}{Structure}
	\begin{itemize}
	\item Enumerators for pieces and info, int array for pegs
	\item Store piece rotation, info, etc. in the enumerator
	\item Store ball/ring, connectors/gaps, in peg info
	\item Handle all GUI-related things in Board, all back-end in LinkGame
	\end{itemize}
\end{frame}
\begin{frame}{Structure}
	% Old class diagram version.
	\fitimagewidth{uml.png}

\end{frame}
\begin{frame}{Structure}
	% Class diagram version.
	%\fitimagewidth{newuml.png}

\end{frame}
\begin{frame}[fragile]
	\frametitle{Piece and Peg classes, mk. I}
	% Note that we ended up rewriting Piece somewhere along the line accidentally instead of just implementing the enumerator.
	\begin{columns}[T]
		\column{0.4\textwidth}
		Enumerator declaration
			\begin{lstlisting}
enum Piece {
A (LINE,OA,
	RING,2,3,
	BALL,4,0,
	BALL,1,0),
...
}
			\end{lstlisting}
		\column{0.67\textwidth}
		Peg declaration
			\begin{lstlisting}
public Pegs(int[] states) {
	this.ballExit = states[0];
	this.ballCon1 = states[1];
	this.ballCon2 = states[2];
	this.ringExit = states[3];
	this.ringOpn1 = states[4];
	this.ringOpn2 = states[5];
}
		\end{lstlisting}
	\end{columns}

\end{frame}
\begin{frame}{Invalid placements}
	% Decisions on why we used snapping or warnings, issues with implementing warnings?
	\begin{itemize}
	\item
	\end{itemize}
\end{frame}
\begin{frame}{Game remake buttons}
	% New game and restart. Decisions, issues (with hints)

\end{frame}
\begin{frame}{Board / LinkGame}

\end{frame}

\section{Screenshots}
\begin{frame} 				\centering{\Huge Screenshots}	\end{frame}
\begin{frame}{Welcome screen}	\fitimageheight{welcome.png}	\end{frame}
%\begin{frame}{Main game}		\fitimageheight{game.png}		\end{frame}
% Moving a piece and hints
%\begin{frame}{Solving}		\fitimageheight{hints.png}	\end{frame}
% Warning
%\begin{frame}{Can't do that}	\fitimageheight{warning.png}	\end{frame}
%\begin{frame}{Victory}		\fitimageheight{victory.png}	\end{frame}

\section{Credits}
\begin{frame}{Credits}

\end{frame}

\appendix

\end{document}